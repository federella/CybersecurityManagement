\section{Conclusions}
	 To conclude this analysis, it is worth explaining why the CIS Critical Security Controls alone are not enough to keep an organization safe. First of all, the document only presents a summary of what controls should be implemented, with little to no details on how it should be done. Moreover, security controls are described in a ``one size fits all" manner, even if the reality strays far from that: each organization is different, depending on the business area they operate in and the type of information they deal with.\\\\
	 This means that a certain level of security expertise is needed to tailor the controls to the organization's needs before implementing them: this should involve a risk assessment process, where security professionals review and prioritize their critical assets and data and decides which security controls are the most relevant to their case.\\\\
	 Finally, it is also important to stress that implementing the controls suggested by CIS or any other information security framework does not automatically means that the organization is secure, it just indicates compliance with said standards. Controls (together with staff!) should be constantly reviewed, updated and tuned in order to keep up with the ever-evolving scenario of threats and cyber attacks.