\section{Ethical Implications}
	The evolution of computer and technology certainly made life easier under several aspects both for individuals and businesses, but it does not come without a price. This is why a new branch of applied ethics was created, called \textit{computer ethics}: the term was conied by Walter Maner in the mid-70s, and it refers to the study of all those ethical problems ``aggravated, transformed or created by computer 
	technology"\cite{cyberethics}.\\\\
	%
	Companies that try to enforce some of the security controls mentioned in the previous section cannot do so without taking into consideration the ethical issues that come with them, which are illustrated in the following sections.
	%
	\subsection{Email and Web Security}
	First and foremost, certain controls might affect the \textit{privacy} of the employees. As discussed before, it might be important to monitor the exchange of emails in order to identify potentially malicious messages. There are several considerations to be made:
	\begin{itemize}
		\item[--] Is it ethical for a company to access employees email, even if it is to avoid loss/theft of sensitive corporate data?
		\item[--] Should the company be able to read the content of the email? Or should they have access only to headers and attachments?
		\item[--] Should employees be allowed to access their personal email account while at work? If so, should the company monitor both personal and professional emails?
		\item[--] Should this policy be disclosed to employees? 
	\end{itemize}
	Similar issues affect the monitoring of web activity through the use of web proxies and such. Although the ultimate goal is to ensure the security of corporate data, the following questions arise:
	\begin{itemize}
		\item[--] Is it ethical for companies to access the web history of their employees?
		\item[--] Should such data be logged? If so, who should be able to access it?
		\item[--] If something problematic were to be found, should management be involved? Even if it means to jeopardize the employee's reputation?
	\end{itemize}
	\subsection{Endpoint Security}
	Employees may also be affected by the adoption of endpoint security solutions. As mentioned before, certain systems may incorporate User Behaviour Analysis: this type of software monitors the user's activity under normal system conditions and keeps track of other information such as date,time and location of login events. This data is then used to generate a profile of what is considered to be a regular set of activities inside the system, and every behaviour that deviates from this will trigger an alert.\\\\
	%
	There are several ethical cosiderations to be made about this approach:
	\begin{itemize}
		\item[--] Should companies use such solutions, even if they might affect the overall performance of the system, causing disturbance or disruption of staff activities?
		\item[--] Is it ethical for companies to use software that performs keystroke analysis to better determine the profile of a user? This actually means having a keylogger installed on the employee machine.
		\item[--] Should companies be able to collect data about user location and date/time of login events? Or is it a violation of their privacy?
	\end{itemize}
	

	\subsection{Security Awareness} 
	The goal of Security Awareness is to provide an insight on what information security is and why it should be considered an integrant part of the business, providing knowledge about corporate policies and regulations at the same time. It should be clear for employees what their jobs and duties are, and most importantly who to contact if they notice something suspicious or potentially malicious.\\\\
	This includes what are commonly called \textit{whistleblowing policies}. The term ``whistleblower" is used to indicate an employee that reports misconduct to people or entities that could take corrective action. Whistleblower policies are needed to make sure that employees have an anonymous way to report illegal practices or violations of corporate policies, without fearing any form of retaliation or discrimination.\\\\
	The topic of whistleblowers is a delicate one, since there is a clear ethical conflict: on one hand, there is loyalty to the employer, while on the other there's loyalty to one's moral principles. This topic becomes even more complicated when there are governative agencies involved or companies dealing with sensitive information or public security: is it worth it to let the truth out, even if it would mean to jeopardize sensitive missions that are aimed at protecting citiziens?
