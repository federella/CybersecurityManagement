\section{The Attack}
	After discussing what weaknesses were affecting Anthem, this section offers an insight on how the attack was carried out and how the aforementioned weaknesses were exploited by the attackers. 
	\subsection{Initial Compromise}
	As discussed in the article, the root cause for the data breach was \textit{spear phishing}. According to the recollection of the events, the attackers gathered as many information as possible about a couple of employees in technical roles (via Facebook, Linkedin and such) and then used said information to craft a legitimate-looking email with a malicious attachment. \\
	The employees were tricked into opening the attachment, consequentially providing the attackers with remote access to their machine.\\\\
	This was possible for two reasons: first of all, employees did not receive a thorough \textit{security awareness training} that would have provided them with the ability to recognize non-legitimate emails and to report them. Secondly, Anthem did not deploy any solution for \textit{email security} on their network, which could have analyzed and detected the malicious attachment before it could reach the user's mailbox.

	\subsection{Privilege Escalation \& Lateral Movement}
	After the first initial compromise, the attacker was able to perform privilege escalation. This suggests that the company did not have any form of \textit{privileged access management} in place, that would have prevented applications from running with administrative privileges.\\\\ %
	%
	The attackers were then able to move laterally and compromise even more accounts. This was probably made possible by exploting vulnerabilities affecting systems on the network. As some audits \cite{anthemAuditReport} showed, Anthem had numerous servers either unpatched or running unsupported operating systems version: this shows a lack of a proper \textit{vulnerability assessment/mitigation process}.\\\\ % 
	%
	It is also interesting to point out that the intrusion was only detected because an employee noticed a query on the database they did not initiate. This shows a lack of several security controls that prevented Anthem from having complete \textit{visibility} over what was happening on their systems.\\\\First of all, even if the company probably did collect logs from several systems on the network, the review process of said logs was not appropriate: logs should be reviewed on a daily basis, either manually or by deploying a Security Information and Event Management (SIEM), in order to identify unusual activities that might mean that the system was compromised.\\\\
	Moreover, the company probably did not deploy any kind of \textit{User Behaviour Analytics} system: this type of software could have picked up abnormal user behaviour (e.g. unusual login time and/or location, unusual activity - compared to what the employee usually does) and trigger an alert.
	%
	%
	\subsection{Data Exfiltration}
	After successfully accessing the database, the attackers reportedly were able to expose over 80 million customer records, completely unnoticed. The attackers were able to do so because Anthem was lacking \textit{Data Loss Prevention (DLP)} controls. This includes monitoring and analyzing the network traffick at egress point near the perementer, in order to detect sensitive or confidential data that is being transferred in violation of security policies.
