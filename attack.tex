\section{The Attack}
	After discussing what weaknesses were affecting Anthem, this section offers an insight on how the attack was carried out and how the aforementioned weknesses were exploited by the attackers. 
	\subsection{Initial Compromise}
	As discussed in the article, the root cause for the data breach is \textit{spear phishing}. According to the recollection of the events, the attackers gathered as many information as possible about a couple of tech employees (via Facebook, Linkedin and such) and then used said information to craft a legitimate-looking email with a malicious attachment. \\
	The employees were tricked into opening the attachment, consequentially providing the attackers with remote access to the machine.\\\\
	This was possible for two reasons: first of all, employees did not receive a thorough \textit{security awareness training} that would have provided them with the ability to recognize non-legitimate emails and to report them. Secondly, Anthem did not deploy any solution for \textit{email security} on their network, which could analyse and detect the malicious attachment before it could reach the user's mailbox.
	%How did this happen? First of all, the users' behaviour shows a lack of \textit{awareness training}. As mentioned before, companies should have an awareness training program in place to educate them about corporate policies, procedures and best practices with regards to information security.\\
	%A thorough awareness training would have provided employees with the ability to recognize a non-legitimate email and report it to the appropriate department, without opening the attachment.\\\\
	%It also appears that Anthem did not have any solutions for \textit{email security} on their network. A Secure Email Gateway, for example, could have provided protection against phishing emails by means of signature-based and sandboxing inspections of the attachment and email authentication methods to detect spoofing.\\\\
	%
	%The malware contained in the email was probably especially crafted for the attack, so even if the company had an \textit{anti-virus} solution in place it may not have been useful, as the malware's signature would probably not match any entry in the signature database. It is not clear whether \textit{two-factor authentication (2FA)} was enabled or not: a lack of 2FA would definitely make it easier to perform such an attack, but it does not really make much of a difference in the scenario presented, since the attacker infected a machine residing on the corporate network.
	%
	\subsection{Privilege Escalation \& Lateral Movement}
	After the first initial compromise, the attacker was able to perform privilege escalation. This suggests that the company did not have any form of \textit{privileged access management} in place, that would have prevented applications to run with administrative privileges.\\\\ %
	%
	The attackers were then able to move laterally and compromise even more accounts. This was probably made possible by exploting vulnerabilities affecting systems on the network. As some audits\cite{anthemAuditReport} showed, Anthem had numerous servers either unpatched or running unsupported operating systems: this shows a lack of a proper \textit{vulnerability assessment/mitigation process}.\\\\ % 
	%
	It is also interesting to point out that the intrusion was only detected because an employee noticed a query on the database they did not initiate. This shows a lack of several security controls that prevented Anthem from having complete \textit{visibility} over what was happening on their systems.\\First of all, even if the company probably did collect logs from several systems on the network, the review process of said logs was not appropriate: logs should be reviewed on a daily basis, either manually or by deploying a Security Information and Event Management (SIEM), in order to identify unusual activities that might mean that the system was compromised.\\\\
	Moreover, the company probably did not deploy any kind of \textit{User Behaviour Analytics} system: this type of software could have picked up abnormal user behaviour (e.g. unusual login time and/or location, unusual activity - compared to what the employee usually does) and trigger an alert.
	%

	%\textit{User Behaviour Analytics (UBA)} software focuses on a range of specific user activities \cite{uba} in order to identify abnormal user behaviour that may indicate that the account was compromised. UBA software usually compares current user activity to their historic activity and the activity of other users in similar roles; moreover, it might check things like the location of the session and the time of the day, even if the credential was authenticated through valid (phished) credentials. This kind of software helps mitigate the risks of both insider and ousider threats.
	%
	\subsection{Data Exfiltration}
	After successfully accessing the database, the attackers reportedly were able to expose over 80 million customer records, completely unnoticed. The attackers were able to do so because Anthem was lacking \textit{Data Loss Prevention (DLP)} controls. This includes traffic monitoring and analysis at egress point near the perementer, to detect sensitive or confidential data that is being sent in violation of security policies.
