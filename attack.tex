\section{The Attack}
	\subsection{Initial Compromise}
	As discussed in the article, the root cause for the data breach is \textit{spear phishing}. According to the reconstruction of the events, the attackers gathered as many information as possible about a couple of tech employees (via Facebook, Linkedin and such) and then used said information to craft a legitimate-looking email with a malicious attachment. \\
	The employees were tricked into opening the attachment, consequentially providing the attackers with remote access to the machine.\\\\
	Why did this happen? First of all, the users' behaviour shows a lack of \textit{awareness training}. As mentioned before, companies should have an awareness training program in place to educate them about corporate policies, procedures and best practices with regards to information security.\\
	A thorough awareness training would have provided employees with the ability to recognize a non-legitimate email and report it to the appropriate department, without opening the attachment.\\\\
	It also appears that Anthem did not have any solutions for \textit{email security} on their network. A Secure Email Gateway, for example, could have provided protection against phishing emails by means of signature-based and sandboxing inspections of the attachment and email authentication methods to detect spoofing.\\\\
	%
	The malware contained in the email was probably especially crafted for the attack, so even if the company had an \textit{anti-virus} solution in place it may not have been useful, as the malware's signature would probably not match any entry in the signature database. It is not clear whether \textit{two-factor authentication (2FA)} was enabled or not: a lack of 2FA would definitely make it easier to perform such an attack, but it does not really make much of a difference in the scenario presented, since the attacker infected a machine residing on the corporate network.
	%
	\subsection{Privilege Escalation \& Lateral Movement}
	After the first initial compromise, the attacker was able to perform privilege escalation. This suggests that the company did not have any form of \textit{privileged access management} in place, that would have prevented applications to run with administrative privileges. Another solution could have been using a \textit{privileged access workstation}, which provides a dedicated and secured operating system to performe sensitive and privileged operations\cite{paws}.\\\\ %
	%
	The attackers were then able to move laterally and compromising more accounts. This was probably made possible by exploting vulnerabilities affecting systems on the network. As some audits\cite{anthemAuditReport} showed, Anthem had numerous servers either unpatched or running unsupported operating systems: this shows a lack of a proper \textit{vulnerability assessment/mitigation process}. Systems should be scanned reguarly and kept up-to-date with security patches; moreover, end-of-life hardware and software should be replaced.\\\\ % 
	%
	It is also interesting to point out that the intrusion was only detected because an employee noticed a query on the database they did not initiate. This shows a lack of several security controls. First of all, the company probably did not collect logs to be used in a \textit{Security Information and Event Management (SIEM)}. SIEMs provide visibility on the whole company by storing, analyzing and correlating different types of security events (authentication events, anti-virus events, intrusion events...). Moreover, SIEMs can be tuned by specifying rules and "normal" behaviour. Any suspicious activity would be promptly picked up, generating an alert requiring immediate action. \\\\
	%
	\textit{User Behaviour Analytics (UBA)} software focuses on a range of specific user activities \cite{uba} in order to identify abnormal user behaviour that may indicate that the account was compromised. UBA software usually compares current user activity to their historic activity and the activity of other users in similar roles; moreover, it might check things like the location of the session and the time of the day, even if the credential was authenticated through valid (phished) credentials. This kind of software helps mitigate the risks of both insider and ousider threats.
	%
	\subsection{Data Exfiltration}
	After successfully accessing the database, the attackers reportedly were able to expose over 80 million customer records, completely unnoticed. The attackers were able to do so because Anthem was lacking \textit{Data Loss Prevention (DLP)} controls. This includes traffic monitoring and analysis at egress point near the perementer, to detect sensitive or confidential data that is being sent in violation of security policies.
