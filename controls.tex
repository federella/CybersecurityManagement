\section{Security Controls}
	Upon reading Steve Ragan's article \cite{breacharticle} describing the nature of the breach that affected Anthem, it appears that the company's security measures were both insufficient and inadequate. This means that it was not only a matter of \textit{which} controls were missing: the ones that were actually in place may have not been configured or even used properly.\\\\
	From what I gathered, the following security measures were either missing or not properly implemented at the time of the attack. 
	\subsection{Email and Web Security}
	Due to the popularity of emails as attack vectors, company should take appropriate measures to mitigate such risks, by using specialized systems as Secure Email Gateway, sophisticated spam filters and so on.\\\\In addition to that, it is often crucial to monitor user activity on the Internet (for example through the use of a web proxy), in order to prevent employees from accessing malicious websites that could compromise their machines.
	\subsection{Endpoint Security \& User Monitoring}
	Endpoint Protection solutions allow enterprises to secure user workstations and prevent even the more sophisticated attacks. This kind of solutions may involve monitoring user activities to define a pattern of user behaviour to detect later anomalies.
	\subsection{Security Awareness Training}
	It is unclear whether Anthem had a Security Awareness program in place for their employees. However, even if there was, it was clearly inadequate. All employees, especially those with access to critical systems, should be educated on corporate policies, procedures and best practices with regards to information security.
	\subsection{Privileged Access Management}
	Administrative privileges on computers, networks and applications should be assigned and managed properly, according to the principle of least privilege. Moreover, the company should have full visibility and control over all privileged accounts across their assets. This serves two purposes, mitigating the risks posed by insider threats and preventing data breaches.
	\subsection{Logging}
	Logging should be enabled on every system for security purposes. Logs should be collected, aggregated and analyzed in order to identify anomalies and abnormal events.
	\subsection{Vulnerability Management}
	Companies should have a thorough vulnerability assessment/management progeam in place. Systems should be scanned regurarly in order to identify, classify and mitigate vulnerabilities.
	\subsection{Data Loss Prevention}
	In order to prevent exfiltration, data should be monitored at all stages: in-use, in-motion and at rest. DLP solutions are focused on preventing unauthorized access, abnormal use and unauthorized copies/leakage.