\section{Security Controls}
	Upon reading Steve Ragan's article \cite{breacharticle} describing the nature of the breach that affected Anthem, it appears that the company's security measures were both insufficient and inadequate. This means that it was not only a matter of \textit{which} controls were missing: the ones that were actually in place may have not been configured or even used properly.\\\\
	From what I gathered, the company was affected by the following weaknesses: 
	\subsection{Lack of Email and Web Security}
	Due to the popularity of emails as attack vectors, companies should take appropriate measures to mitigate such risks, by using specialized systems as Secure Email Gateway, sophisticated spam filters and so on. This would have provided protection against phishing emails by means of signature-based and sandboxing inspections of the attachment and email authentication methods to detect spoofing\\\\In addition to that, it is crucial to monitor user activity on the Internet (for example through the use of a web proxy), in order to prevent employees from accessing malicious websites that could compromise their machines.
	\subsection{Insufficient Endpoint Security}
	Endpoint Protection solutions allow enterprises to secure user workstations and prevent even the more sophisticated attacks. This kind of solution may involve monitoring user activities to define a pattern of ``normal" user behaviour to detect anomalies. Other solutions may include Host Intrusion Detection and Prevention Systems (HIPS/HIDS) and next-generation antiviruses.
	\subsection{Inadequate Security Awareness Training}
	It is unclear whether Anthem had a Security Awareness program in place for their employees. However, even if there was, it was clearly inadequate. All employees, especially those with access to critical systems, should be educated on corporate policies, procedures and best practices with regards to information security.
	\subsection{No Management of Administrative Privileges}
	Administrative privileges on computers, networks and applications should be assigned and managed properly, according to the principle of least privilege (meaning that employees should be given access to the minimum set of information needed to perform their job properly). Moreover, the company should have full visibility and control over all privileged accounts across their assets, which serves two purposes: mitigating the risks posed by insider threats and preventing data breaches.
	\subsection{Inadequate Review of Logs}
	Logging should be enabled on every system for security purposes, especially on critical assets containing sensitive business data. Logs should be collected, aggregated and analyzed in order to identify anomalies and abnormal events. Review of logs could allow security professional to detect intrusions and unauthorized access.
	\subsection{No Vulnerability Management Process}
	Companies should have a thorough vulnerability assessment/management program in place. Systems should be scanned regurarly in order to identify, classify and mitigate vulnerabilities. Moreover, operating systems and applications should be kept up-to-date with security patches and updates.
	\subsection{Lack of Data Loss Prevention}
	In order to prevent exfiltration, data should be monitored at all stages: in-use, in-motion and at rest. DLP solutions are focused on preventing unauthorized access, abnormal use and unauthorized copies/leakage: these solutions may include next-generation firewalls, e-mail gateways, web proxies and so on.