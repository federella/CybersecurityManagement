	\section{Introduction}
	The goal of this assignment is to discuss the data breach that affected Anthem, the second largest health insurer in the United States, in 2014. The report will focus on how the attack was carried out, what weaknesses were exploited and what should have been done to prevent and mitigate it.\\\\The documents is structured as follows:
	\begin{itemize}
		\item[--] The \textit{second} paragraph contains an analysis on the control weaknesses that exposed the organization to the attack, as described in the article ``Anthem: How does a breach like this happen?"\cite{breacharticle}. For each of the identified controls, there is a short description and an explanation of why they should have been implemented.
		\item[--] The \textit{third} paragraph focuses on the topic of computer ethics. Ethical issues will be presented with reference to the control weaknesses identified in the previous section, explaining how and why they might affect the company.
		\item[--] The \textit{fourth} paragraph offers a reconstruction of the life cycle of the attack, starting from the initial compromise all the way through the exfiltration of stolen data. This will be done by referencing the weaknesses discussed in section two, explaining how the attackers exploited them and what the consequences were.
		\item[--] The \textit{fifth} paragraph is be centered around how (and why) following information security standars/best practices would have mitigated the vulnerabilities that caused the company to be attacked. The document I chose for this purpose was the ``Top 20 Critical Controls" by the Center for Internet Security (CIS): starting from this standard, I chose six different controls and discussed their relationship to the weaknesses that affected Anthem.
		\item[--] The \textit{sixth} paragraph is dedicated to a conclusive analysis of the potential limitations of the CIS controls that were illustrated in the previous section of the document, explaining why they are not enough to ensure security.
	\end{itemize}
